\begin{titlepage} 
\maketitle
\vspace*{\fill}
Status: Draft revision
Version: 0.5
Student number: 452434
Degree: Software Engineering
Saxion University of Applied Sciences
Bachelor Thesis
\thispagestyle{empty}
\end{titlepage}

\chapter*{Contact information}
\begin{center}
\begin{tabular}{ | m{8em} | m{8em} | m{8em} | m{11.1em} | }
\hline
\textbf{Name} & \textbf{Organization} & \textbf{Function} & \textbf{E-mail}
\\ \hline
Thymo van Beers & BauWatch Technology Group & Graduate student & thymovanbeers@gmail.com
\\ \hline
Wouter Horlings & BauWatch Technology Group & Company mentor & whorlings@bouwatch.nl
\\ \hline
Peter Ebben & Saxion & Saxion mentor & p.w.g.ebben@saxion.nl
\\ \hline
\end{tabular}
\end{center}
\thispagestyle{empty}
%\chapter*{Acknowledgements}
%TBD
%\thispagestyle{empty}
\chapter*{Summary}
As a part of the graduation phase of the bachelor degree HBO-ICT: Software Engineering this thesis has been written by Thymo van Beers at BauWatch during a research project lasting 20 weeks.
BauWatch has a large amount of cameras in operation throughout Europe.
Currently they have no ability to adequately manage the parameters the cameras are configured with.
The main research question is, "How can a large number of remote security cameras of different models and brands be managed from a centralized system?"
The goal of the project was to research possible solutions and to build a prototype which allowed for the centralized remote management of camera parameters.

In order to define the requirements of the system a focus group was formed with two engineers who would further develop on the project outcomes and an interview was conducted with product management to better define the problems being experienced.

With the obtained input research was done and an idea was lended from other lifecycle management software to orchestrate camera paramaters into templates.
Additionally a method is given for maintaining revisions of these templates using git.
These techniques have been combined into a system design and the workings of this design are implemented as a prototype.

The results of this thesis indicate that management of different cameras through a single system is possible and that the design allows for the addition of extra camera models and parameter types without much effort.

In the future the prototype could be extended with a graphical user interface and a client/server model as the current solution only provides for single user access through a text interface.
\thispagestyle{empty}

\thispagestyle{empty}
\tableofcontents
\thispagestyle{empty}

\chapter{Introduction}

\section{Context}
BauWatch (known as BouWatch in Dutch) is a security system supplier that develops and operates temporary video surveillance and access control systems for use at construction sites and other remote locations.
One of their most popular products are the video masts (called a "BauWatch"), a portable, self-contained, metal enclosure including an internal
power source, network connectivity and a telescopic mast with cameras, green floodlights and other peripherals. A BauWatch can be lifted by a forklift and
easily transported to the next location. Research \& Development for these products is done by BauWatch Technology Group (BTG) located in Enschede.

\section{Company history}
BauWatch Technology Group has its roots at the Ericsson research facility that was housed in Enschede until 2003. During the 90s and early 2000s Ericsson was doing
research on DECT, Bluetooth, 3G and WiFi. After Ericsson closed their Enschede facility, the Twente Institute for Wireless and Mobile Communications (TI-WMC) was created as a technology R\&D spin-off specialized in wireless mobile communications systems. In 2007 the result of various networking related projects were combined in the FIGO product line that combined Wi-Fi-based multi-radio meshing with cellular and Ethernet connectivity for public safety applications\cite{noauthor_twente_nodate}. The name FIGO comes from the Italian word meaning: fig, cool, tasteful or sexy.

In 2011 FIGO was split-off to be a sister organization to TI-WMC to focus on the sales and further development of the FIGO product.
After TI-WMC defaulted in 2014, FIGO continued alone at a new location where it still operates from today. 
The years after the split, FIGO did various contracted projects like Sensor City Assen, Bad Boys Buster with Luminoxx, and BauWatch.
After various projects with BauWatch, FIGO was acquired in 2018 to become BauWatch' dedicated R\&D department known as BauWatch Technology Group.
The focus was no longer solely on networking equipment but to also include work integrating the cameras and Network Video Recorders (NVR) being used by BauWatch.

\section{Problem statement}
% Er zijn teveel handelingen/ om een camera te configuren.
% Momenteel is er geen inzicht/controle over welke versie van een configuratie een camera heeft.
% Daardoor mogen een aantal instellingen niet zomaar aangepast worden als dit voor een situatie wenselijk is.
% Versiebeheer

% At the moment BauWatch does not have a system to remotely administer its cameras. This makes it difficult to gain insight into the current state of the cameras and make an adjustment to a range of cameras to make a correction. Because of this BauWatch is forced to forbid changes to settings even though making a change to a setting is sometimes desired.


% At the moment BauWatch does not have an efficient method to remotely configure its cameras. The current way of working is to configure a camera in a
% certain way with a group of "base settings". These settings are then exported to a configuration file that can be imported to the other cameras.
% When these settings need to be changed, a new configuration file can be made with the desired changes. Because these configuration files do not
% carry any
% information about versioning or changes, it is very hard to keep track of which configuration a camera is using. 

% For example, it could be desired to let the camera at a certain location be more sensitive to movement in order to trigger an alarm. In order to do this the alarm sensitivity
% setting would need to be altered to best suit that location. A problem arises when the camera is moved to a new location where there is more
% movement. This would cause an excessive amount of alarms to be generated, meaning an increased workload for the dispatchers. In order to avoid this problem, settings like this are fixed to a certain value to prevent excessive alarms but still be sensitive enough to cause detection.

%%%%%%%

% BauWatch wants to be able to remotely configure cameras in an efficient manner. This way they can have greater flexibility with the parameters being used
% by the cameras, so they can quickly make adjustments when required. An example of an adjustment could be to change the cameras alarm sensitivity to movement. Right now there is no way to easily administer these settings across these cameras. Without a way to administer this a camera might not work properly if it is moved to
% a location it wasn't configured for. In order to avoid this problem, settings are set to a default value which have been agreed to not be changed to optimize performance. These settings are distributed through the proprietary configuration backups from a camera. This makes it hard to keep track of which version of the configuration is being used.
BauWatch has more than 10,000 cameras deployed on its videos masts throughout Europe. At the moment there is no adequate method to manage these cameras.
Without a proper management solution it is impossible to get an overview of what configuration a camera is using.
The current workflow is to configure the cameras once after delivery from the manufacturer.
After this initial configuration the configuration is not allowed to be changed.

Nevertheless there still are situations where an alternate configuration is desired as it increases the surveillance quality.
In some situations this alternate configuration is applied despite the fact that this goes against the established workflow.
When a camera is moved to a different location there is no automated method to verify if a configuration has been altered or not.
Therefore a camera with an alternative configuration can be moved to a different location without restoring the configuration to its original state.
This causes the camera to no longer operate as expected.
The only way it is currently possible to verify a configuration is by checking the settings manually or restoring from a backup configuration file.


\section{Objective}
BauWatch wants to find a solution which makes it possible to manage camera configurations.
The condition to this solution is that the alternate configurations can be remotely administered in an easy way.
This should be done in such a way that its insightful what configuration a camera is using.
The objective is to build a prototype of a software system that is able to configure the cameras to a predefined set of parameters while still allowing
more fine grained control for a specific subset of those cameras.

In order to determine how this system should be built, a research question was formulated and divided in subquestions. They will be described in the next section.

% De wens van BauWatch is om het mogelijk te maken om alternatieve configuraties te gebruiken wanneer deze de kwaliteit van de bewaking te verbeteren. Onder de voorwaarde dat het mogelijk om deze alternatieve configuraties te kunnen beheren.
% Hierbij kan gedacht worden aan het bijo
% Dit moet gedaan worden op een manier dat het wel inzichtelijk blijft welke camera over welke configuratie beschikt.

% Onderzoek hoe een camera geconfigureerd kan worden op basis van een template zodat deze centraal uitgerold kan worden en er detectie mogelijk is op het moment dat een camera niet meer aan het template voldoet. Verder is het belangrijk dat er een vorm van versiebeheer aan deze configuraties kan worden verbonden zodat alle (of een subselectie) van camera's teruggezet kunnen worden naar een eerder versie van een template indien blijkt dat er een wijziging is gemaakt die problemen geeft.

% BauWatch wants to find a solution to this problem so that cameras can be remotely configured from a predefined template while also allowing specific parameters to be overridden for individual cameras.
% To solve this problem a software system will be developed that allows the user to create a template that can be
% applied to a video mast. This template can be modified by the user by interacting with a central server through a
% web interface. Templates can be stacked in multiple layers such that a user can make a descendant template for a specific
% situation. After the user applies the template to a range of cameras, the central server will convert this template to a set of API commands understood by each camera and will then push those commands to those devices. In addition to
% these templates, the system should have the capability
% to track who made a change to a template by keeping a revision
% history with the changes that were made. That way a template can be restored to an earlier version in case a change turns out to cause a problem.
% Cameras will need to be registered with the system using their model number, IP-address and authentication information since there is no software system or database from which this can be queried. The system should also have the ability to detect any mismatches between a template and camera so the user knows that it needs to be updated.

\section{Main question}
The main question that the project will answer is: ,,How can a large number of remote security cameras of different models and brands be managed from a centralized system?"
This question is further divided into the following sub-questions.

\section{Subquestions}
In order to come to an answer to the main research question it has been split into the following subquestions.
These questions will be answered in this thesis using methods from the Development Oriented Triangulation (DOT) framework \cite{noauthor_dot_nodate}.
\begin{enumerate}
	\item What parameters are of interest to BauWatch?
	\item How should a configuration be represented to express settings for different cameras?
	\item How would the system architecture of a centralized parameter system be designed and implemented?
\end{enumerate}

\section{Scope}
% The project has commence on 15-11-2021 and will be completed by 22-04-2022. The scope of the project will be limited to the configuration of cameras through their APIs only. No modifications 
% will be done to the firmware of cameras and no special consideration is given to BauWatch' network architecture other than the cameras being reachable on a unique IP-address. Furthermore the focus of the prototype will be with the manufacture API instead of ONVIF. ONVIF will only be used
% in the specific case a part of the manufacturer API is found not to be working properly or if implementing a call to the API would cause an unacceptable delay.
% If during the project it is determined that the implementation of the prototype will finish early, an addition can be made to add the ability for equipment other than cameras (speakers,
% lights, etc.) to be configured.
% The design for the prototype will not add any specifics regarding other equipment but will be designed in such a way that this addition can be made without much effort if required.

% The first camera that will be investigated is a Hikvision DS-2DE4225IW-DE. BauWatch expressed that these cameras will be more actively used in the future and that it would be nice to have a working prototype using this camera model. The second camera that will be investigated is a VCA IPX3802SV because it
% provides a different API from the Hikvision and is actively in use within BauWatch.

The project commenced on November 15th 2021 and ended on April 22nd 2022. The scope of the project was limited to the configuration of cameras through their own APIs only and
without making any firmware modifications. Since only a proof of concept was to be made, no special consideration to BauWatch' network architecture was given although it became
evident that this would not cause any obstacles either since all cameras can be reached on a unique IP-address.
%During the definition phase of the project a thesis on implementing
%an ONVIF plugin was found. To prevent overlap the project was reformulated so that ONVIF would not be required, the prototype could however support an interface to ONVIF in the
%future if such an interface is deemed necessary.

Two cameras were used for the project. Namely, a Hikvision DS-2DE4225IW-DE and a VCA IPX3802SV.
The selection for these cameras was made based on the fact that their APIs are designed in completely different ways and that they are already in use or expected to be used more in the future.
The only thing the APIs of both cameras have in common is their use of HTTP.
