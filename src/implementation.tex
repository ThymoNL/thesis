\chapter{Implementation}
\section{Development environment}
BauWatch has git repository templates for commonly used programming languages.
These templates contain a standardized directory structure along with configuration files for the Gitlab code pipelines.
This makes it easy to start a project and immediately get started with a basic environment to run linters, documentation generators and automatic tests.

\section{Modules}
Module based programming worth describing?
Maybe interesting to describe the source layout?
Built using Makefile

The application is divided into three modules and one main file.

\subsection{Continuous Integration}
By default all repositories are set up to check all code submitted as a merge request complies with a set of merge checks.
First of all the continuous integration environment checks if the code successully compiles and passes its checks.
After that the commits are checked to verify they refer to the Jira story the feature branch is associated with and if a changelog file describing the changes for that story is present.
If all checks pass the merge request will be marked with a green check mark indicating it is ready to be reviewed by someone.
Once all review comments have been resolved and the reviewer approves the changes they can be merged back into the master branch by someone with maintainer permissions for that repository.

%TODO Maybe add something about the tools used? These are gitlab (CI), GNU Make, golang tools, Docker. Might be useful to describe testing


%\section{Parameters not in parent}
%Parameters can optionally be overridden by the base templates children. This is implemented using a map.
%An issue arises when in child 2 param c is added. That param is not defined in child 1 or the base template.
%How can this param be set in the first place?
%Guard against this in the model and disallow adding parameters through the webpanel.
%Additionally the base template should be initialized from some presupplied default, be it yaml or constants.

\section{Template implementation}
Pics:
Struct, interactions between templates
Describe how template stacking works

\section{Using the system}
The system can be used from the commandline using the following commands:
\begin{enumerate}
\item help: List the available commands
\item exit: Exit the program
\item quit: Alias for exit
\item camera: Configure a camera
\item template: Configure a template
\item save: Save a template revision
\item load: Load a template from disk
\item compare: Check if a camera is configured according to its template and if not show the differences
\item configure: Configure a camera using the template that has been assigned to it
\end{enumerate}

TODO Show how to do each step.

\section{Saving}
A lot of time was initially spent implementing the template save and loading functionality.
First an attempt was done using polymorphism of yaml. However this proved problematic as during unmarshaling the librarby did not know what to instantiate.
After trying to fix this a different approach was taken that ended up being a lot simpler.

\section{Loading}
When a load instruction is given to the system a call is done to LoadTemplate with the name of the template to be loaded.
This function loads the requested template and all other templates needed to satisfy its parents.
If a template could not be found this function will return nothing and the program will print an informational message that the template could not be loaded.
If the template depends on a parent template and this template could not be found an error condition is reported.
This is done because the program only allows a template deletion when it does not have any other templates that depend on it.
Thus this condition can only if the templates stored on disk were corrupted or modified outside the program.
To limit the scope of the prototype this condition can not be recovered from and has to be manually resolved by removing the affected templates or editing the YAML files to get them back in the proper state.

\section{Hikvision}
The Hikvision api was implemented using the Hikvision ISTAPI documentation\cite{noauthor_intelligent_nodate}.
Two paramaters had to be implemented.
The first one was the motion detection sensitivity paramater.
The XML for this has been mapped in a struct and is configured by reading the object, editing it and writing it back.
Since the whole object is used it is trivial to implement more of these settings.
For the purposes of this prototype only the sensitiviy parameter was used.
TBD: Insert example XML from camera
%TODO insert dummy XML payload

In order to compare the changes to the template the API implementation uses the three steps as follows.
First the prepare step must be executed whenever an action using the API is to be done.
This has been split since both the compare and the configure step would have to do the same actions of getting the information from the camera bbefore doing further operations.
After the prepare step has run the results are stored in the APIData field of the camera being operated on for use by the other functions.
A choice can be made to do either the compare or the configure step.
When a compare is done the XML is examined and their representations as paramaeters is constructed.
These values are then compared with the values from the template to detect any mismatches.
If a mismatch has been detected the parameter is added to a list so the program can output the expected and the actual value of that parameter.

The configure step also depends on the data gathered using the prepare step and modifies the downloaded XML with the values from the template.
By using the GetParam function of a template the template will provide a pointer to the proper parameter as described in the parameter enumeration part.
The parameter provides the value that should be input to the XML body and when all paramaters that the API supports have been updated the configuration is written back to the camera as documented in the manufacturer API documentation.

If the camera reports an error condition during the configuration process this error condition is reported backt to the user.

\section{VCA}
At the time of writing an implementation that can interact with a VCA camera has not been completed yet.
This will be part of the final portfolio.
