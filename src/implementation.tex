\chapter{Implementation}
\section{Development environment}
BauWatch has git repository templates for commonly used programming languages.
These templates contain a standardized directory structure along with configuration files for the Gitlab code pipelines.
This makes it easy to start a project and immediately get started with a basic environment to run linters, documentation generators and automatic tests.

\subsection{Continuous Integration}
By default all repositories are set up to check all code submitted as a merge request complies with a set of merge checks.
First of all the continuous integration environment checks if the code successully compiles and passes its checks.
After that the commits are checked to verify they refer to the Jira story the feature branch is associated with and if a changelog file describing the changes for that story is present.
If all checks pass the merge request will be marked with a green check mark indicating it is ready to be reviewed by someone.
Once all review comments have been resolved and the reviewer approves the changes they can be merged back into the master branch by someone with maintainer permissions for that repository.

%TODO Maybe add something about the tools used? These are gitlab (CI), GNU Make, golang tools, Docker. Might be useful to describe testing


%\section{Parameters not in parent}
%Parameters can optionally be overridden by the base templates children. This is implemented using a map.
%An issue arises when in child 2 param c is added. That param is not defined in child 1 or the base template.
%How can this param be set in the first place?
%Guard against this in the model and disallow adding parameters through the webpanel.
%Additionally the base template should be initialized from some presupplied default, be it yaml or constants.

\section{Template implementation}
Pics:
Struct, interactions between templates
Describe how template stacking works
