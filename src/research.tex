% Bauwatch heeft verspreid door EU meer dan 10k cameras in het veld. Momenteel is er geen toereikende methode om al deze cameras te beheren.
% Zonder een toereikende beheeroplossing is het niet mogelijk om een goed overzicht te houden van welke camera over welke instellingen beschikt.
% De huidige werkwijze is om de camera's vanuit de fabriek eenmaal met de juiste instellingen te configureren.
% Na deze eerste configuratie is het niet toegestaan om deze instellingen aan te passen.

% Desondanks zijn er situaties waar een alternatieve configuratie gewenst is omdat het de kwaliteit van de bewaking verhoogt. In sommige
% situaties wordt deze alternatieve configuratie wel toegepast ookal is de werkwijze om dit niet te doen.
% Wanneer een camera naar een andere locatie wordt verplaatst is er geen manier om in te zien of de configuratie nog klopt.
% Hierdoor kan een camera met een specifieke configuratie op een andere locatie terecht komen.
% Het gevolg hiervan is dat de camer niet meer naar verwachting werkt.

%Peter:
% How does it work now?
% 



\section{What parameters are of interest to BauWatch?}
Cameras include a lot of settings that could be relevant to BauWatch.
It's good to know what parameters should be implemented into the prototype and what parameters are still relevant, but don't directly add value to the prototype as their implementation would be very similar to others.
In order to determine the parameters that are relevant to BauWatch an interview was conducted on January 11th 2022 with Alex van der Leij, a product manager responsible for the video masts deployed around Europe.

\section{How would the system architecture of a centralized parameter system be designed and implemented?}
Once it is known how a configuration can be represented this should then be applied to the design of the prototype. This design will then be implemented into a
prototype to demonstrate the concept using a couple of parameters that BauWatch deems relevant.

% While preparations were being made for this project some preliminary library research has been conducted. During this period a previous thesis on creating an ,,ONVIF Plug-in" \cite{tim_k_onvif} was found. This thesis was found to contain a lot of material that would have been produced in the original scope of the project. This caused some
% delays during the definition phase of the project as the scope of the assignment had to be slightly adjusted to prevent doing redundant work. 
% The new project that was defined aims to improve areas where ONVIF does not apply. Cameras do not always implement all components through ONVIF
% or if they do they may not properly follow the specification. These cameras do however have their own API that is known to operate properly.
