\chapter{Research}

\section{Previous Work}
BTG already did some experimentation to better automate equipment configuration by creating some scripts that replayed API commands but this is still
in a very rudimentary state. Each time a setting needed to be applied to a large amount of cameras, a request would be made to an engineer who would then craft a script
specifically for that situation.

%%% Interview with Alex, where to put this?

% TODO: Add reference to chapter explaining solution!!!

% Indicated that:
% Takes a lot of work to manage cameras
% The ability/flexibility to configure cameras is not sufficient
% Improvements can be made
% Its difficult to maintain an overview of what templates are being used
% There is not version control
% Current tooling can not work with batches
% Rec: Look at Hikvision Batch Configuration

% What parameters could be relevant?
% Autofocus/zoom
% Video settings (e.g. Bitrate)
% Netwerk settings (suggested, but out of scope)
% Detection sensitivity, to limit alarms sent to the dispatches
% Would be nice to have the ability to quickly change these (overnight) to balance alarm frequency and flase alarms

% Concern about mapping settings between cameras
% Idea with Wouter was to allow this to be configured inside the API.+ Not part of interview though.
% Example map mismatch: Field of view differences between VCA and Hikvision
% TODO Design phase not clear
\subsection{Interview}
The purpose of the interview was to gain a better understanding of the problems experienced
by the product organisation in the context of configuring cameras and to elicit requirements for the prototype. During the interview he indicated that it currently takes a lot of work to manage
the cameras and that the current method of configuring cameras is not flexible enough which could be improved upon. He explained that the way the cameras are currently being
administered is to configure a camera with a set of default settings which are to be considered a set of 'base settings'. These can then be exported to a file which is used as a
'template' for other
cameras. A downside of this approach is that it is difficult to maintain different versions of these files and that it can't easily be determined what version a camera is using.
Another problem he indicated was that current tooling does not provide a practical way to make a single change to a batch of cameras.


After introducing the goal of the project the question was asked as to what parameters he would find valuable to have in the initial prototype.
The parameters that were named as candidates were: automatic camera focus and zoom, video quality, network configuration and detection sensitivity.
Out of these parameters some of them were of particular interest to him. The first one being video quality. It would be nice to configure video quality related settings like a
cameras bitrate and resolution because video masts may be in locations with reduced cellular network connectivity. In that case it would be beneficial to lower the video quality
for these cameras to reduce network load. The second parameter of interest was detection sensitivity. This parameter regulates how sensitive a camera is to movement before it
triggers an alarm to the dispatchers to attract their attention. It would be nice to quickly change this parameter so that the frequency of alarms and false alarms can be balanced.


At the end of the interview some concerns were raised by Alex about how settings might be implemented differently by camera manufacturers. For example a sensitivity setting of 10
on camera A might not cause the same behavior if the same value is used for camera B. A solution to this problem should be found and is described in chapter \ref{sec:design}.

\section{How should a configuration be represented to express settings for different cameras?}
The camera configurations should be represented in a generic way so that they can be applied to different cameras. There also needs to be a way to detect if a camera
has gone out of sync with the stored configuration and a way to revert configurations to previous revisions in case an erroneous setting was applied. To make this comparison the assumption is made that settings can be read back from the cameras using their APIs.

%TODO: Maybe add screenshots here?
\section{Available product analysis}
Before the design and implementation of the product were made, an analysis on already available products was done to see if there were any existing solutions that could be used to satisfy the objective or provide a partial solution to the problem.
During the analysis no solutions were found that directly addressed the problem of configuring a large amount of cameras in a generic way.
However a solution was found that implemented a templating system aimed at configuring Unix-like and Windows servers.
This solution is called Foreman.

,,Foreman is an open source project that helps system administrators manage servers throughout their lifecycle, from provisioning and configuration to orchestration and monitoring." \cite{noauthor_foreman_nodate}
The project gives the administrator the ability to automate repetitive tasks by supportingan interface with common configuration mannagement solutions like Puppet, Ansible, Chef and Salt.

In addition to being a frontend to these configuration management solutions Foreman also provides a host management system known as ,,Host Groups" \cite{noauthor_foreman_nodate-1}. In this context a host is any Linux client that Foreman manages.
A host group functions as a template for common host settings.
Hosts that have been assigned to a host group inherit any settings that have been defined in that group.
Host groups can also be nested into a hierarchy.
This way the topmost host group can serve as a base level group containing general settings for all hosts.
The host groups that have been nested under this group will inherit its settings but also allow the administrator to provide more specific settings for that group.
For example one could have a base level host group that defines the settings for the operating system, and nested host groups can define settings for the software running on that system.

While Foreman uses these host groups for defining parameters for operating systems this concept also translates to other types of systems.
Because of that the choice is made to use these host groups as the basis for designing the templating system for the cameras.

TODO: Design after this chapter. This chapter introduces the template concept.
