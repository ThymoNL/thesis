\documentclass[11pt,a4paper,nomath,nopackage]{siltex-report}
\usepackage{amsmath}
\usepackage[nomath]{siltex}
\usepackage[language=int]{bauwatchtitle}
\usepackage{graphics}
\usepackage{pdflscape}
\usepackage{multirow}
\usepackage{tabulary}
\usepackage{float}
\usepackage{listings}
\usepackage[acronym]{glossaries}
\usepackage[style=ieee]{biblatex}
\usepackage{svg}
%\usepackage[s]{hyperref}

\usepackage{zref-savepos}

\addbibresource{refs.bib}

\definecolor{black}{rgb}{0,0,0}

\graphicspath{ {./img/} }

\newcommand\YAMLcolonstyle{\color{red}\mdseries}
\newcommand\YAMLkeystyle{\color{black}\bfseries}
\newcommand\YAMLvaluestyle{\color{blue}\mdseries}

\makeatletter

% here is a macro expanding to the name of the language
% (handy if you decide to change it further down the road)
\newcommand\language@yaml{yaml}

\expandafter\expandafter\expandafter\lstdefinelanguage
\expandafter{\language@yaml}
{
  keywords={true,false,null,y,n},
  keywordstyle=\color{darkgray}\bfseries,
  basicstyle=\YAMLkeystyle,                                 % assuming a key comes first
  sensitive=false,
  comment=[l]{\#},
  morecomment=[s]{/*}{*/},
  commentstyle=\color{purple}\ttfamily,
  stringstyle=\YAMLvaluestyle\ttfamily,
  moredelim=[l][\color{orange}]{\&},
  moredelim=[l][\color{magenta}]{*},
  moredelim=**[il][\YAMLcolonstyle{:}\YAMLvaluestyle]{:},   % switch to value style at :
  morestring=[b]',
  morestring=[b]",
  literate =    {---}{{\ProcessThreeDashes}}3
                {>}{{\textcolor{red}\textgreater}}1     
                {|}{{\textcolor{red}\textbar}}1 
                {\ -\ }{{\mdseries\ -\ }}3,
}

% switch to key style at EOL
\lst@AddToHook{EveryLine}{\ifx\lst@language\language@yaml\YAMLkeystyle\fi}
\makeatother

\newcommand\ProcessThreeDashes{\llap{\color{cyan}\mdseries-{-}-}}


\newcounter{NoTableEntry}
\renewcommand*{\theNoTableEntry}{NTE-\the\value{NoTableEntry}}

\newcommand*{\notableentry}{%
  \multicolumn{1}{@{}c@{}|}{%
    \stepcounter{NoTableEntry}%
    \vadjust pre{\zsavepos{\theNoTableEntry t}}% top
    \vadjust{\zsavepos{\theNoTableEntry b}}% bottom
    \zsavepos{\theNoTableEntry l}% left
    \hspace{0pt plus 1filll}%
    \zsavepos{\theNoTableEntry r}% right
    \tikz[overlay]{%
      \draw[red]
        let
          \n{llx}={\zposx{\theNoTableEntry l}sp-\zposx{\theNoTableEntry r}sp},
          \n{urx}={0},
          \n{lly}={\zposy{\theNoTableEntry b}sp-\zposy{\theNoTableEntry r}sp},
          \n{ury}={\zposy{\theNoTableEntry t}sp-\zposy{\theNoTableEntry r}sp}
        in
        (\n{llx}, \n{lly}) -- (\n{urx}, \n{ury})
        (\n{llx}, \n{ury}) -- (\n{urx}, \n{lly})
      ;
    }% 
  }%
}

\DefineBibliographyStrings{english}{%
	bibliography = {References},
}

\hyphenation{Bau-Watch}

%%%%%%%%%%%%% Content below %%%%%%%%%%%%%

\title{Camera Provisioning}
\author{Thymo van Beers}
\date{February 2022}
\extrainfo{
	Bachelor thesis
	\newline
	HBO-ICT: Software Engineering
	\newline
	Saxion University of Applied Sciences
}

\begin{document}

\begin{titlepage} 
\maketitle
\vspace*{\fill}
Status: Final\newline
Version: 1.0\newline
Student number: 452434\newline
Degree: Software Engineering\newline
Saxion University of Applied Sciences\newline
Bachelor Thesis
\thispagestyle{empty}
\end{titlepage}

\chapter*{Contact information}
\begin{center}
\begin{tabular}{ | m{8em} | m{8em} | m{8em} | m{11.1em} | }
\hline
\textbf{Name} & \textbf{Organization} & \textbf{Function} & \textbf{E-mail}
\\ \hline
Thymo van Beers & BauWatch Technology Group & Graduate student & thymovanbeers@gmail.com
\\ \hline
Wouter Horlings & BauWatch Technology Group & Company mentor & whorlings@bouwatch.nl
\\ \hline
Peter Ebben & Saxion & Saxion mentor & p.w.g.ebben@saxion.nl
\\ \hline
\end{tabular}
\end{center}

\thispagestyle{empty}
\chapter*{Summary}
This thesis is the result of the graduation phase of the HBO-ICT: Software Engineering bachelor degree by Thymo van Beers.
A research project was conducted at BauWatch lasting 20 weeks.
Bauwatch provides outdoor security throughout Europe using thousands of cameras.
The goal of this thesis is to design a solution that can easily manage these cameras.
The main research question is, "How can a large number of remote security cameras of different models and brands be managed from a centralized system?"
The goal of the project was to research possible solutions and to build a prototype which allowed for the centralized remote management of camera parameters.

In order to define the requirements of the systen a focus group was formed with two engineers.
They would provide some technical requirements so the project outcomes could be used to develop a product after the project ends.
Additionally, an interview was conducted with product management to better define the problems being experienced.

Using these requirements a proof of concept of a centralized system has been developed that stores configuration parameters as templates.
The configuration using these templates can be done through an extensible interface so that support for new types of devices can easily be added.
Additionally a method is given for maintaining revisions of these templates using git.
These techniques have been combined into a system design and the workings of this design are implemented as a prototype.

The results of this thesis show that management of different cameras through a single system is possible.
In the future the prototype could be extended with a graphical user interface and a client/server model as the current solution only provides for single user access through a text interface.
\thispagestyle{empty}

\thispagestyle{empty}
\tableofcontents
\thispagestyle{empty}

\chapter{Introduction}

\section{Context}
BauWatch (known as BouWatch in Dutch) is a security system supplier that develops and operates temporary video surveillance and access control systems for use at construction sites and other remote locations.
One of their most popular products are the video masts (called a "BauWatch"), a portable, self-contained, metal enclosure including an internal
power source, network connectivity and a telescopic mast with cameras, green floodlights and other peripherals. A BauWatch can be lifted by a forklift and
easily transported to the next location. Research \& Development for these products is done by BauWatch Technology Group (BTG) located in Enschede.

\section{Company history}
BauWatch Technology Group has its roots at the Ericsson research facility that was housed in Enschede until 2003. During the 90s and early 2000s Ericsson was doing
research on DECT, Bluetooth, 3G and WiFi. After Ericsson closed their Enschede facility, the Twente Institute for Wireless and Mobile Communications (TI-WMC) was created as a technology R\&D spin-off specialized in wireless mobile communications systems. In 2007 the result of various networking related projects were combined in the FIGO product line that combined Wi-Fi-based multi-radio meshing with cellular and Ethernet connectivity for public safety applications\cite{noauthor_twente_nodate}. The name FIGO comes from the Italian word meaning: fig, cool, tasteful or sexy.

In 2011 FIGO was split-off to be a sister organization to TI-WMC to focus on the sales and further development of the FIGO product.
After TI-WMC defaulted in 2014, FIGO continued alone at a new location where it still operates from today. 
The years after the split, FIGO did various contracted projects like Sensor City Assen, Bad Boys Buster with Luminoxx, and BauWatch.
After various projects with BauWatch, FIGO was acquired in 2018 to become BauWatch' dedicated R\&D department known as BauWatch Technology Group.
The focus was no longer solely on networking equipment but to also include work integrating the cameras and Network Video Recorders (NVR) being used by BauWatch.

\section{Problem statement}
% Er zijn teveel handelingen/ om een camera te configuren.
% Momenteel is er geen inzicht/controle over welke versie van een configuratie een camera heeft.
% Daardoor mogen een aantal instellingen niet zomaar aangepast worden als dit voor een situatie wenselijk is.
% Versiebeheer

% At the moment BauWatch does not have a system to remotely administer its cameras. This makes it difficult to gain insight into the current state of the cameras and make an adjustment to a range of cameras to make a correction. Because of this BauWatch is forced to forbid changes to settings even though making a change to a setting is sometimes desired.


% At the moment BauWatch does not have an efficient method to remotely configure its cameras. The current way of working is to configure a camera in a
% certain way with a group of "base settings". These settings are then exported to a configuration file that can be imported to the other cameras.
% When these settings need to be changed, a new configuration file can be made with the desired changes. Because these configuration files do not
% carry any
% information about versioning or changes, it is very hard to keep track of which configuration a camera is using. 

% For example, it could be desired to let the camera at a certain location be more sensitive to movement in order to trigger an alarm. In order to do this the alarm sensitivity
% setting would need to be altered to best suit that location. A problem arises when the camera is moved to a new location where there is more
% movement. This would cause an excessive amount of alarms to be generated, meaning an increased workload for the dispatchers. In order to avoid this problem, settings like this are fixed to a certain value to prevent excessive alarms but still be sensitive enough to cause detection.

%%%%%%%

% BauWatch wants to be able to remotely configure cameras in an efficient manner. This way they can have greater flexibility with the parameters being used
% by the cameras, so they can quickly make adjustments when required. An example of an adjustment could be to change the cameras alarm sensitivity to movement. Right now there is no way to easily administer these settings across these cameras. Without a way to administer this a camera might not work properly if it is moved to
% a location it wasn't configured for. In order to avoid this problem, settings are set to a default value which have been agreed to not be changed to optimize performance. These settings are distributed through the proprietary configuration backups from a camera. This makes it hard to keep track of which version of the configuration is being used.
BauWatch has more than 10,000 cameras deployed on its videos masts throughout Europe. At the moment there is no adequate method to manage these cameras.
Without a proper management solution it is impossible to get an overview of what configuration a camera is using.
The current workflow is to configure the cameras once after delivery from the manufacturer.
After this initial configuration the configuration is not allowed to be changed.

Nevertheless there still are situations where an alternate configuration is desired as it increases the surveillance quality.
In some situations this alternate configuration is applied despite the fact that this goes against the established workflow.
When a camera is moved to a different location there is no automated method to verify if a configuration has been altered or not.
Therefore a camera with an alternative configuration can be moved to a different location without restoring the configuration to its original state.
This causes the camera to no longer operate as expected.
The only way it is currently possible to verify a configuration is by checking the settings manually or restoring from a backup configuration file.


\section{Objective}
BauWatch wants to find a solution which makes it possible to manage camera configurations.
The condition to this solution is that the alternate configurations can be remotely administered in an easy way.
This should be done in such a way that it's insightful what configuration a camera is using.
The objective is to build a prototype of a software system that is able to configure the cameras to a predefined set of parameters while still allowing
more fine grained control for a specific subset of those cameras.

In order to determine how this system should be built, a research question was formulated and divided in subquestions. They will be described in the next sections.

% De wens van BauWatch is om het mogelijk te maken om alternatieve configuraties te gebruiken wanneer deze de kwaliteit van de bewaking te verbeteren. Onder de voorwaarde dat het mogelijk om deze alternatieve configuraties te kunnen beheren.
% Hierbij kan gedacht worden aan het bijo
% Dit moet gedaan worden op een manier dat het wel inzichtelijk blijft welke camera over welke configuratie beschikt.

% Onderzoek hoe een camera geconfigureerd kan worden op basis van een template zodat deze centraal uitgerold kan worden en er detectie mogelijk is op het moment dat een camera niet meer aan het template voldoet. Verder is het belangrijk dat er een vorm van versiebeheer aan deze configuraties kan worden verbonden zodat alle (of een subselectie) van camera's teruggezet kunnen worden naar een eerder versie van een template indien blijkt dat er een wijziging is gemaakt die problemen geeft.

% BauWatch wants to find a solution to this problem so that cameras can be remotely configured from a predefined template while also allowing specific parameters to be overridden for individual cameras.
% To solve this problem a software system will be developed that allows the user to create a template that can be
% applied to a video mast. This template can be modified by the user by interacting with a central server through a
% web interface. Templates can be stacked in multiple layers such that a user can make a descendant template for a specific
% situation. After the user applies the template to a range of cameras, the central server will convert this template to a set of API commands understood by each camera and will then push those commands to those devices. In addition to
% these templates, the system should have the capability
% to track who made a change to a template by keeping a revision
% history with the changes that were made. That way a template can be restored to an earlier version in case a change turns out to cause a problem.
% Cameras will need to be registered with the system using their model number, IP-address and authentication information since there is no software system or database from which this can be queried. The system should also have the ability to detect any mismatches between a template and camera so the user knows that it needs to be updated.

\section{Main question}
The main question that this thesis will answer is: ,,How can a large number of remote security cameras of different models and brands be managed from a centralized system?"
This question is further divided into the sub-questions listed in the next section.

\section{Subquestions}
In order to come to an answer to the main research question it has been split into the following subquestions.
These questions will be answered in this thesis using methods from the Development Oriented Triangulation (DOT) framework \cite{noauthor_dot_nodate}.
\begin{itemize}
	\item What parameters are of interest to BauWatch?
	\item How would the system architecture of a centralized parameter system be designed and implemented?
	\item How should a configuration be represented to express settings for different cameras?
\end{itemize}

\section{Scope}
% The project has commence on 15-11-2021 and will be completed by 22-04-2022. The scope of the project will be limited to the configuration of cameras through their APIs only. No modifications 
% will be done to the firmware of cameras and no special consideration is given to BauWatch' network architecture other than the cameras being reachable on a unique IP-address. Furthermore the focus of the prototype will be with the manufacture API instead of ONVIF. ONVIF will only be used
% in the specific case a part of the manufacturer API is found not to be working properly or if implementing a call to the API would cause an unacceptable delay.
% If during the project it is determined that the implementation of the prototype will finish early, an addition can be made to add the ability for equipment other than cameras (speakers,
% lights, etc.) to be configured.
% The design for the prototype will not add any specifics regarding other equipment but will be designed in such a way that this addition can be made without much effort if required.

% The first camera that will be investigated is a Hikvision DS-2DE4225IW-DE. BauWatch expressed that these cameras will be more actively used in the future and that it would be nice to have a working prototype using this camera model. The second camera that will be investigated is a VCA IPX3802SV because it
% provides a different API from the Hikvision and is actively in use within BauWatch.

The project commenced on November 15th 2021 and ended on April 22nd 2022. The scope of the project was limited to the configuration of cameras through their manufacturer specific APIs only and without making any firmware modifications.
The network architecture within BauWatch is specifically out of scope of this thesis, as it is possible to reach any camera on there specific DNS address.

%Since only a proof of concept was to be made, no special consideration to BauWatch' network architecture was given although it became evident that this would not cause any obstacles either since all cameras can be reached on a unique IP-address.
%During the definition phase of the project a thesis on implementing
%an ONVIF plugin was found. To prevent overlap the project was reformulated so that ONVIF would not be required, the prototype could however support an interface to ONVIF in the
%future if such an interface is deemed necessary.

Two cameras were used for the project. Namely, a Hikvision DS-2DE4225IW-DE and a VCA IPX3802SV.
The selection for these cameras was made based on the fact that their APIs are designed in completely different ways and that they are already in use or expected to be used more in the future.
The only thing the APIs of both cameras have in common is their use of HTTP.

\chapter{Research}

\section{Previous work}
Some experimentation was already done by BTG before the start of the project that attempted to automate the manual process of configuring a camera.
This only goes as far as writing a script for a specific situation that makes an API call for a specific setting so multiple cameras of the same type can be changed in one go.
While this makes it a little less time consuming to change a setting across all cameras this does not solve the problem of the lack of an overview of the current configuration nor is it user-friendly requiring an engineer to write a script each time a setting needs to be changed.

\section{Focus group}
To better define what the project deliverables would be, a focus group was formed consisting of the graduate student and  Wouter Horlings and Silke Hofstra, who are software engineers at BTG.
During the group sessions it was determined that a system should be implemented capable of configuring a camera in such a way that the configuration does not explicitly need to specify what model of camera is being configured.
The assumption can be made for this project that both cameras are similar in operation but different in configuration but it would be nice if the system could be easily extended to support devices with different capabilities.
This could be done by separating the configuration parameters into generic options and letting the camera specific configuration code determine what options are needed for that particular camera to configure it according to those options.
The focus group also determined that it is desired to be able to rollback a camera configuration using some sort of version control system in case a change to the configuration presented problems.

% Indicated that:
% Takes a lot of work to manage cameras
% The ability/flexibility to configure cameras is not sufficient
% Improvements can be made
% Its difficult to maintain an overview of what templates are being used
% There is not version control
% Current tooling can not work with batches
% Rec: Look at Hikvision Batch Configuration

% What parameters could be relevant?
% Autofocus/zoom
% Video settings (e.g. Bitrate)
% Netwerk settings (suggested, but out of scope)
% Detection sensitivity, to limit alarms sent to the dispatches
% Would be nice to have the ability to quickly change these (overnight) to balance alarm frequency and flase alarms

% Concern about mapping settings between cameras
% Idea with Wouter was to allow this to be configured inside the API.+ Not part of interview though.
% Example map mismatch: Field of view differences between VCA and Hikvision
% TODO Design phase not clear
\section{Interview}
To gain a better insight of the camera configuration related problems experienced by the product organization an interview was conducted with Alex van der Leij, a product manager responsible for the video masts deployed around Europe.
During the interview he indicated that it currently takes a lot of work to manage
the cameras and that the current method of configuring cameras is not flexible enough which could be improved upon. He explained that the way the cameras are currently being
administered is to configure a camera with a set of default settings which are to be considered a set of 'base settings'. These can then be exported to a file which is used as a
'template' for other
cameras. A downside of this approach is that it is difficult to maintain different versions of these files and that it can't easily be determined what version a camera is using.
Another problem he indicated was that current tooling does not provide a practical way to make a single change to a batch of cameras.


After introducing the goal of the project the question was asked as to what parameters he would find valuable to have in the initial prototype.
The parameters that were named as candidates were: automatic camera focus and zoom, video quality, network configuration and detection sensitivity.
Out of these parameters, some were of particular interest to him. The first one being video quality. It would be nice to configure video quality related settings like a
camera's bitrate and resolution because video masts may be in locations with reduced cellular network connectivity. In that case it would be beneficial to lower the video quality
for these cameras to reduce network load. The second parameter of interest was detection sensitivity. This parameter regulates how sensitive a camera is to movement before it
triggers an alarm to the dispatchers to attract their attention. It would be nice to quickly change this parameter so that the frequency of alarms and false alarms can be balanced.


At the end of the interview some concerns were raised by Alex about how settings might be implemented differently by camera manufacturers. For example a sensitivity setting of 10
on camera A might not cause the same behavior if the same value is used for camera B. A solution to this problem should be found and is described in chapter \ref{sec:design}.

%\section{How should a configuration be represented to express settings for different cameras?}
%The camera configurations should be represented in a generic way so that they can be applied to different cameras. There also needs to be a way to detect if a camera
%has gone out of sync with the stored configuration and a way to revert configurations to previous revisions in case an erroneous setting was applied. To make this comparison the assumption is made that settings can be read back from the cameras using their APIs.

\section{Available product analysis}
Before the design and implementation of the prototype were made, an analysis on already available products was done to see if there were any existing solutions that could be used to satisfy the objective or provide a partial solution to the problem.

As part of the analysis an online search was conducted using search terms like "camera lifecycle management software", "camera configuration tool" and "universal ip camera configuration tool".
This search led to results of some tools that could discover IP cameras using protocols like Bonjour, ONVIF Device Discovery or UPnP but none of these tools provided a capability to configure a camera as can be done with the built-in browser based interface of the camera or the manufacturer provided API.
Because no existing solution satisfied the needs of the stakeholders the search was broadened to other kinds of lifecycle management software to see if some of the concepts could be applied to a new system aimed at configuring cameras. 
A solution was found that implemented a templating system aimed at configuring Unix-like and Windows servers.
This solution is called Foreman.

,,Foreman is an open source project that helps system administrators manage servers throughout their lifecycle, from provisioning and configuration to orchestration and monitoring." \cite{noauthor_foreman_nodate}
The project gives the administrator the ability to automate repetitive tasks by supporting an interface with common configuration management solutions like Puppet, Ansible, Chef and Salt.

In addition to being a frontend to these configuration management solutions Foreman also provides a host management system known as ,,Host Groups" \cite{noauthor_foreman_nodate-1}. In this context a host is any Linux client that Foreman manages.
A host group functions as a template for common host settings.
Hosts that have been assigned to a host group inherit any settings that have been defined in that group.
Host groups can also be nested into a hierarchy.
This way the topmost host group can serve as a base level group containing general settings for all hosts.
The host groups that have been nested under this group will inherit its settings but also allow the administrator to provide more specific settings for that group.
For example one could have a base level host group that defines the settings for the operating system, and nested host groups can define settings for the software running on that system.

While Foreman uses these host groups for defining parameters for operating systems this concept also translates to other types of systems.
Because of that the choice is made to use the philosophy of host groups as the basis for designing the templating system for the cameras.

\section{Revision history}
As was indicated by the focus group there should be a way to keep a revision history between templates.
When a change to a template turns out to be problematic it should be possible to revert it to an earlier revision.
Furthermore it should also be possible to know who made a change and with what intention.
To determine the best way to implement this a couple of options were evaluated.

\subsection{One file per revision}
The first option that was evaluated was to use one file per template revision.
Using this method templates would be stored in a directory named after the template.
Inside that directory a template is stored as a file named with the revision number and the file extension.
This way the latest revision can be found by sorting the files by name in descending order.
When a template is reverted to a previous version the files could be sorted in the same manner and will be deleted or otherwise marked inactive until the revision to be reverted to is reached. An example of how this directory structure would be stored is seen in figure \ref{fig:diskstruct}

\begin{figure}[h!]
	\centering
	\includegraphics[scale=0.5]{yaml_dir_struct}
	\caption{Directory structure of templates and their revisions using one file per revision.}
	\label{fig:diskstruct}
\end{figure}

Storing templates in this manner has one flaw.
Namely that of sorting by name when more than 10 revisions have been made.
When sorting the files by name the tenth revision would be sorted between files starting with 1 and 2.
In this case the descending sorted order would then be: 9, 8, 7, 6, 5, 4, 3, 2, 10, 1.
This could be prevented by adding leading zeroes to the filename but that only delays the problem until the amount of revisions overflows the leading zeroes.
If for example four digits would be used where the unused digits are zeroed the descending sorted order would be: 0010, 0009, 0008, 0007, 0006, 0005, 0004, 0003, 0002, 0001.
As can be seen this would solve the problem but as soon as more than 9999 revisions are made the same problem occurs.
Another potential solution unaffected by this problem is the use of the file modification time as stored on the filesystem.
Using this file attribute would work in most cases although it can be affected by external factors such as changing system time or restoring from backups where the proper file attributes are not restored \cite{noauthor_mtime_2018}.

\subsection{Git}
The second option that was evaluated was to use git.
While git is most often used for tracking changes in software source files its manpage describes it as a fast, scalable, distributed revision control system or "stupid" content tracker, since it makes no assumption about what content is stored in it \cite{truyers_git_2016}.

Git can be used to keep track of changes in a repository by chaining them into commits.
Each commit contains a snapshot of a set of files, a message that gives a description about the changes and metadata like who made a change and when.
When a new commit is made a reference to that commit is stored that tells you what the latest commit in the chain is.
This reference is called the HEAD and can be seen in figure \ref{fig:gitcommit}.
By changing what commit the HEAD points to you can revert the state of the files in your repository back to an earlier commit.

\begin{marginfigure}
	\centering
	\includegraphics{git_commit}
	\caption{The HEAD is moved to the latest commit.}
	\label{fig:gitcommit}
\end{marginfigure}

Normally a user of git would interact with the repository through a command line or graphical user interface using a set of commands.
There are many different commands that are available to the user.
Examples of the most commonly used commands to view and record changes are: `git add` to mark files to be committed (called staging), `git commit` to commit all the files that have been staged, `git push` to upload local changes to a remote git repository and `git status` to get a brief summary of that state of the repository. There are many more commands that can be used but for the purpose of brevity these have been omitted.

Several libraries exist that implement these commands and allow us to use git from a program to make commits of our data and to revert back to an earlier version if necessary.
Some examples are libgit2\footnote{\url{https://github.com/libgit2/libgit2}}, a pure C based implementation, and go-git\footnote{\url{https://pkg.go.dev/github.com/go-git/go-git/v5}}, an implementation in the Go programming language.

Git makes a distinction between two types of commands.
The first type of command categorizes the most commonly used high-level commands which have been described earlier and are called "porcelain" functions.
The second type involves the low-level functions that directly modify git's internal datastructure and are called "plumbing" functions.
These plumbing functions are normally used by the "porcelain" functions to implement their behavior but they can also be used for more advanced usecases where the porcelain functions are not sufficient.

An example of such an advanced usecase is to use git as a "bare" repository and only making use of the plumbing fuctions.
A "bare" repostiory in the context of git is a git repository that only contains the \textit{.git} directory without having a branch actively checked out in the working directory.
The method proposed by Truyers\cite{truyers_git_2016} uses the plumbing functions to directly create blobs in git's database, essentially turning it into a NoSQL database.
The claimed advantages of this method are that data only needs to be saved a single time by creating a blob in the database without having to write it to disk first.
In addition this method is claimed to benefit from automatic data deduplication provided by git and an ability exists to work on multiple branches at the same time without the need of having multiple checked out directories.

Keeping in consideration the objective of the project it was determined that the offline storage of data would not necessitate the use of multiple branches.
Furthermore the benefit of data deduplication does not apply because there is no expectation that any duplicate files will be stored.
This means only the benefits of saving data a single time in the git database remain.
Because of this it was decided that the extra time needed to implement the proposed method would not outweigh the limited benefits that would be gained by using it over the regular method using porcelain functions.

\subsection{Conclusion}
Looking at both of these options it is evident that the manual approach of using numbered files has some undesirable effects.
Because of that the choice is made to implement revisions using git's porcelain functions.
Considering that only a prototype will be built that uses just a handful of templates the benefits of using Truyers' method are considered to be negligible and would not outweigh the time needed to implement it.
If in the future the method using porcelain functions proves to be a performance- or storage bottleneck Truyers' method can be revisited to further evaluate its use to optimize the storage of data.

\chapter{Design}
\label{sec:design}
In this chapter the design for the Camera Provisioning system will be laid out.
The goal of the system is to provide a solution which makes it possible to manage the configuration of cameras.
The system allows the user to orchestrate parameters into different templates which can be assigned to a camera.
The design contains both functional and technical specifications containing the descriptions of different components, requirements and supplemental diagrams and wireframes.
Before further introducing the design two important disticions need to be made.
A \textit{template} refers to a set of data containing the parameters a camera should be configured with.
A \textit{camera configuration} refers to the information that is required to interface with a camera such as its hostname and login credentials.

\section{Requirements}
In table \ref{tab:requirements} the requirements for the system have been defined.
These requirements have been collected by discussing with the developers from the focus group and conducting an interview with a product manager during the analysis phase.
When deemed necessary the requirements have been finetuned during the development of the prototype, after consulting with the stakeholders, to better explain their goal or when a better solution was found that did not comply with the initial formulation of that requirement.

The requirements are tracked using an identifier starting with a letter followed by a number and are listed in the ID column of table \ref{tab:requirements}.
If the identifier starts with the letter 'B' or a 'U' they specify a business or user requirement respectively.
If the identifier starts with a 'F' or 'NF' they specify a functional or non-functional system requirement respectively.
Requirements have been prioritized using the MoSCoW method \cite{noauthor_moscow_nodate}.

\begin{table*}[h]
    \centering
    \begin{tabulary}{\linewidth}{CLL}
        \textbf{ID} & \textbf{Requirement} & \textbf{MoSCoW}
    %%% Business requirements
        \\ \hline
        B1 & A camera can be configured by the user without detailed technical knowledge about the model being used & Must
        \\ \hline
		% TODO: When is a rev incompatible?
        B2 & A camera configuration can be rolled back to any compatible version in history & Must
        \\ \hline
        B3 & The configuration changes must include auditing showing at least the author and timestamp & Must
        \\ \hline
        B4 & Parameters whose values behave differently across brands shall be scaled by an adjustable factor for each brand & Should
        \\ \hline
        B5 & The application has the option to show the user the differences between the config in the camera and the template & Must
    %%% User requirements
        \\ \hline
        U1 & The interface must only allow authorized users, preferably via username and password & Could
        \\ \hline
        U2 & A user can change his credentials through the interface & Could
        \\ \hline
        U3 & The template used for configuring cameras must include one setting based on a string and one setting based on an integer value & Should
        \\ \hline
        U4 & New settings can be added to the application without having to change the templates & Must
        \\ \hline
        U5 & The application should be able to handle one or more settings in a template that are not present in any of its parents & Should
        \\ \hline
        U6 & The user should be able to change templates via a user interface & Should
        \\ \hline
        U7 & A user can create a new template with a specified parent template & Must
        \\ \hline
		U8 & A user shall be able to trigger the configuration process on one or all cameras according to their template & Must
		\\ \hline
		U9 & The user should be able to configure cameras via a user interface & Should
		\\ \hline
		U10 & The user must be able to assign which template a camera is associated with & Must
    %%% System requirements
        \\ \hline
        NF1 & The application must communicate with the cameras using HTTP & Must
        \\ \hline
        NF2 & The codebase must be checked by the linter as configured by BTG & Must
        \\ \hline
        NF3 & The codebase shall be strongly typed & Should
        \\ \hline
        NF4 & 90\% of code that does not interface with external systems shall be covered by unit tests& Must
		\\ \hline
		NF5 & Code that can not reasonably be covered by unit tests must be covered by manual system tests & Must
        \\ \hline
        NF6 & Templates and camera configuration are stored on disk in a human-readable/writeable format & Must
        \\ \hline
        NF7 & The application is to be exposed as a REST API and documented via OpenAPI & Could
		\\ \hline
		NF8 & The application can be used interactively & Must
        \\ \hline
        NF9 & The templates must be camera independent & Must
    \end{tabulary}
    \caption{Requirements}
    \label{tab:requirements}
\end{table*}

% TODO: Maybe this fits better in implementation since it goes into programming lanuguage, linting etc.
\section{General design}
At the conclusion of this project the prototype is handed over to the back-end team of BTG.
Since they will be further developing the prototype into a production ready system the choice was made to implement the system in Go as that programming language is their preferred language for new products fulfilling requirement NF-2.

\section{Templates}
The prototype will have a user interface through which a template can be configured.
A template in the context of this system is a collection of lists containing camera configuration options.
Templates can be built up from different layers where each layer can override parameters from the layer above.
This allows the user to define a template that is shared by a large amount of cameras but still allows fine-grained control over a smaller subset when this is desired.
These cameras would still use all the settings from the parent template but would have a different setting for the parameter being overridden.


The parameters within a template may or may not be used by the camera configuration API.
It is up to the programmer to decide what parameters a configuration should or should not support.
For example, given two cameras of which one supports motion detection and the other does not supporting motion detection of any kind.
In this case it is known by the camera configuration component that the camera supports motion detection and it shall look for paramaters that define motion detection settings
\footnote{One should note that within the scope of this prototype only the motion detection sensitivity paramater has been implemented but in a production implementation there will be more.}.
The camera configration component also knows that the other camera does not support motion detection and hence will not look in the template for parameters describing motion detection but will limit itself to parameters relevant to that camera.
This allows the template to describe paramaters for multiple types of devices and leaving the responsibility of determining what parameters to use or not use to the components that know how to configure that particular device.
In the future this could also be used to extend the template with parameters for devices other than cameras, provided the system is changed to support the configuration of such devices.

\section{Datastructure}
Templates are represented in memory as a tree using dynamically allocated nodes that have pointers to one parent and zero or more children.
There is a single template that is considered to be the root node of the tree.
All templates that are added to the system will be a child of this root node.
The choice to represent the collection of templates as a tree is to solve the requirement of parameter inheritance.
Using this datastructure is an efficient method to enumerate the values of a template's parameters at configuration time.

A parameter can be enumerated by traversing each parent node starting from the node representing the current template and checking if it defines the sought parameter.
This process is repeated until a value for a given parameter is found.
Figure \ref{fig:find_param} shows how parameters are inherited.
If ParamB would be enumerated at the Child2 node the system would determine that it is not defined in Child2 by checking if it is present in the list of paramaters contained in the node.
The system would then visit the parent nodes until a definition of ParamB is found.
Parameters A and C can be enumerated instantly since they are defined in Child2 itself where ParamC is newly introduced and ParamA overrides the values from Child1 and BaseTemplate.
\begin{marginfigure}
	\centering
	\includegraphics[width=3cm]{find_param}
	\caption{Parameter enumeration}
	\label{fig:find_param}
\end{marginfigure}
The best-case performance for a parameter lookup is O(1) in the case the parameter is already defined in the current template.
The worst-case performance of a lookup is O(n) where n is the number of templates to be traversed from the current node to the root node.

When a new template is instantiated, by creating a new one or loading one from YAML, that template has to be inserted at the correct place inside the tree.
The way this is achieved is to do a breadth-first search (BFS) (Figure \ref{fig:bfs}) in the tree where a match is identified using the name of the parent node.
Using BFS to find the parent node is more efficient than a depth-first search (DFS) as a template is more likely to inherit from a more generic template describing e.g. parameters for a country than a more specific template representing parameters for a specific building site.
\begin{marginfigure}
	\centering
	\includesvg{Sorted_binary_tree_breadth-first_traversal}
	\caption{Sorted binary tree with first nine letters of alphabet, showing inorder traversal\cite{pluke_sorted_2010}}
	\label{fig:bfs}
\end{marginfigure}

\section{Multiple base templates} %TODO: Root template requires more intro?
While this design requires a single root template that all other templates inherit from, the design still allows the creation of multiple trees with independent parameter sets as can be seen in figure \ref{fig:multiple_root}.
This can be done by not setting any parameters in the root template such that the templates directly below it don't have any parameters that can be inherited from the root template.
The children of the root template then become the first template to define the template effectively turning into root templates themselves while still being children of the actual root template.
\begin{marginfigure}
	\centering
	\includegraphics{multiple_root}
	\caption{Defining multiple base templates}
	\label{fig:multiple_root}
\end{marginfigure}

\section{Parameters}
Cameras can contain many different types of parameters.
These parameters can consist of something simple like an integer or a text string but they can also be represented as something more complex like a bitmap representing a section of a rectangle as is done by the Hikvision manufacturer API for example.
 To allow a template to work with these parameters in a generic way some abstractions can be applied.
Templates contain a map of which the key is a string representing the name of a parameter and the value is a pointer to an object implementing the parameter interface.
This way the template class can expose access to its parameters so that the CameraAPI can use them without needing to concern itself with what the type of its parameters are.


The Get and Set method of the Parameter interface behave differently depending on the instance of a parameter.
As can be seen in figure \ref{fig:class_diagram} there are three parameters implementing the parameter interface.

\begin{figure}[h!]
	\centering
	\includegraphics[scale=1]{class_diagram}
	\caption{Class diagram}
	\label{fig:class_diagram}
\end{figure}

% TODO: Explain in detail how each parameter works and how ex. levels are defined.
These are the classes IntegerParameter, StringParameter and LevelParameter.
For example, the IntegerParameter will return an integer when its Get method is called while a StringParameter would return a string.
Go allows for this behavior by specifying the return value to be an \textit{interface\{\}}.
This indicates that the type being returned is an interface that does not necessarily implement any methods and is roughly equivalent in use to the Object class in Java.

% Creating parameters
To instantiate a new paramater a function is provided based on the Factory Method design pattern, the function implementing this factory is called \textit{NewParameter}.
Using this function the instantiation of the proper parameter type can be deferred to the factory so that the code needing to instantiate a paramater does not have to find out which parameter constructor to call.
The \textit{NewParameter} factory function will return a pointer to an object implementing the requested parameter type which can then be used to Get and Set the parameter values.

\section{Storing to file}
Templates are stored on disk as YAML formatted data. YAML was chosen because it is a plain text format that can be interpreted by humans while still being able to describe all attributes that make up a template. 
Due to the plain text nature this also allows the usage of tools like GNU diff to produce a view of the differences between two templates.
An example of a YAML-formatted template can be found in listing \ref{lst:template}.
The following data is used to describe a template in YAML:
\begin{itemize}
	\item name: The name of the template.
	\item parent: The name of the parent template this template should inherit from.
	\item parameters: A list of all the parameters contained in this template.
	\item type: A name that defines the type of a parameter.
	\item name: (under parameters): The name of the parameter.
	\item attributes: A generic list with data used to construct a parameter of the type in the `type` key.
\end{itemize}
\lstinputlisting[language=yaml,caption={Example template YAML},label={lst:template}]{format.yaml}

\section{Reading from file}
Templates are stored on disk with the name of its parent template.
When this file is loaded to instantiate a template the system must be able to discover its parent by its name.
This can be achieved using a breadth-first search of the template tree or by doing a lookup in a hashmap containing a reference to the templates.
%TODO What if a template has not been loaded yet? Keep in a temporary buffer or look for a particular filename
The operation of loading a template and linking these to their parent is described in figure \ref{fig:inserttemplate}.

First the template is read from a file inside the \emph{templates} directory.
This directory is located in the data directory of the program as configured in the \emph{dataDir} variable.
The template file bears the name of the template ending in \emph{.yml} to indicate that the file contains YAML formatted text data.
After the file has been read it is unmarshaled from YAML into an intermediate structure in memory representing the YAML formatted data.
% TODO: Figure that shows structs?
This intermediate structure can then be used to instantiate the final template structure by creating a new template with the name of the template as obtained from the YAML data.
If the template has a parent template it is inserted at the proper place in the tree by doing a lookup of the parent template by name and setting up the links between the nodes.
When the parent template can not be found it is loaded from disk before continuing.
If it does not have a parent the template is marked as the top-level root template serving as a base template for its children.
When a base template already exists an error state exists since there can only be a single root node of the template tree.
This error is reported to the user and the template is discarded from memory.
After this is done the parameters are transferred from the intermediate structure into the final structure.

\begin{figure}[h!]
	\centering
	\includegraphics[scale=0.5]{insert_template_acty}
	\caption{Template insert operation}
	\label{fig:inserttemplate}
\end{figure}

The prototype will load all templates it can find from the template directory once a load operation is requested.
In the future this could be enhanced by lazily loading a template only when a template refers to it but within the demonstrative scope of the prototype the amount of templates will never raise to a number where this optimization would show a measurable difference in performance or memory usage.

\section{Keeping revision history}
In order to fulfill the ability to restore a template to a previous version a revision history is kept.


Design:
Each time a template is saved it is saved as a yaml file and committed in git.
A commit only modifies one file at a time.
A template can be reverted to an earlier commit.
Author information is entered at the login screen.
library available that implements porcelain functions:
https://github.com/go-git/go-git

Decision that still needs to be made:
Keep all templates in the same directory or keep them in separate ones.
Separate directories may be slightly more efficient due to the way git stores these but this may not matter at this scale nor make sense as there is only a single file for a template now.
It could be stored in the same directory.

\section{Camera interface}
The camera interface is a system component that sits between the template component and the cameras themselves. The purpose of this component
is to translate the settings from a template into an API call understood by the camera. Each camera can have a different API with potentially different interpretation of parameters.
While two camera's of the same brand may use the same API.
There still might be some slight differences between two models that are not completely compatible.
Instead of adding a new API that is tailored specifically to that model a provision is made to store information about the camera model inside the Camera struct.
This information could be used by the API to modify it's behavior for that particular model that still allows the use of the other functions of the API that are common to all models of the particular camera vendor.
This component maps the parameter from the template to an appropriate value for that camera.
In order for a camera to accept new parameters they require a form of authentication. All cameras being used to implement the prototype support both HTTP Basic and HTTP Digest authentication.
To successfully interface with the cameras one or both of these authentication methods must be supported by the camera interface component.

TODO Add activity diagram from notes here

\section{Translating generic parameters to camera specific values}
As an example of translating a generic parameter to a camera specific value the detection sensitivity parameter is used.
For the Hikvision camera this is a value that can be configured between 0 and 100 in increments of 20.
For the VCA camera this works by setting a number between 1 and 128.

Because these parameters may behave differently between camera manufacturers and individual models the template parameter is divided into a couple of presets so these presets describing expected behavior.
For example, the presets for sensitivity used in the prototype are labeled as low, medium, high and ultra.
These presets can then be used by the Camera API to replace it with a value that best corresponds the expected behavior.
This way cameras may not have the same value for that parameter but they will behave in the same manner as closely as possible.
These values are implemented in such a way that extra levels can be easily added by the programmer and their values adjusted to fine tune a cameras behavior.
At the time of writing it is not possible to change the defined levels while the program is running but this may be implemented before the presentation at which time details will be explained.

\section{Detecting differences between a template and actual configuration}
As stated in requirement B5 a capability must exit to detect if a camera still is still configured according to the parameters in the configuration.
To achieve this the CameraAPI has been split into three operations that split a configuration action into multiple steps.
The first operation is called Prepare, it pulls the current configuration from the camera and stores it for use by the other steps.
The Prepare operation must be executed each time an operation is to be done to a camera to ensure the systems has the latest configuration from the camera.
The other two operations are Configure and Compare.
The Compare operation uses the parameters from the template and compares these with the parameters from the camera.
If a mismatch exists the parameter being compared is added to a list which is displayed to the user to inform them of the actual and expected values of a parameter.
The Configure operation enumerates all parameters in the template associated with the camera and writes this configuration to the camera.
This way the CameraAPI can be run up to the point that it is ready to write the new parameters to the camera.
At that point it can be checked if any changes would be made and if so what parameters are affected.
The user can then be presented with both the parameter value in the template as the one that is read from the camera.

\section{User interface}
The components of the system can be managed through a web interface.
They will allow the user to create a new template and add parameters to it as seen in figure \ref{fig:templatewireframe}.
The user should also be able to manage the relations between templates as well as registering new cameras to be managed by the system (TBD still).
\begin{figure}[h!]
	\centering
	\includegraphics[scale=0.2]{wireframes}
	\caption{Template overview  wireframe}
	\label{fig:templatewireframe}
\end{figure}

\section{REST API}
Originally the prototype was meant to have an API so that the user could interact with the system through a web interface.
Unfortunately there was not enough time to implement this into the final prototype but some designs for how this API would operate have been made nonetheless.
The API has been described according to the OpenAPI specification.TBD yaml in appendix or PDF somehow.

\chapter{Implementation}
\section{Development environment}
BauWatch has git repository templates for commonly used programming languages.
These templates contain a standardized directory structure along with configuration files for the Gitlab code pipelines.
This makes it easy to start a project and immediately get started with a basic environment to run linters, documentation generators and automatic tests.

\section{Modules}
Module based programming worth describing?
Maybe interesting to describe the source layout?
Built using Makefile

The application is divided into three modules and one main file.

\subsection{Continuous Integration}
By default all repositories are set up to check all code submitted as a merge request complies with a set of merge checks.
First of all the continuous integration environment checks if the code successully compiles and passes its checks.
After that the commits are checked to verify they refer to the Jira story the feature branch is associated with and if a changelog file describing the changes for that story is present.
If all checks pass the merge request will be marked with a green check mark indicating it is ready to be reviewed by someone.
Once all review comments have been resolved and the reviewer approves the changes they can be merged back into the master branch by someone with maintainer permissions for that repository.

%TODO Maybe add something about the tools used? These are gitlab (CI), GNU Make, golang tools, Docker. Might be useful to describe testing


%\section{Parameters not in parent}
%Parameters can optionally be overridden by the base templates children. This is implemented using a map.
%An issue arises when in child 2 param c is added. That param is not defined in child 1 or the base template.
%How can this param be set in the first place?
%Guard against this in the model and disallow adding parameters through the webpanel.
%Additionally the base template should be initialized from some presupplied default, be it yaml or constants.

\section{Template implementation}
Pics:
Struct, interactions between templates
Describe how template stacking works

\section{Using the system}
The system can be used from the commandline using the following commands:
\begin{enumerate}
\item help: List the available commands
\item exit: Exit the program
\item quit: Alias for exit
\item camera: Configure a camera
\item template: Configure a template
\item save: Save a template revision
\item load: Load a template from disk
\item compare: Check if a camera is configured according to its template and if not show the differences
\item configure: Configure a camera using the template that has been assigned to it
\end{enumerate}

TODO Show how to do each step.

\section{Saving}
A lot of time was initially spent implementing the template save and loading functionality.
First an attempt was done using polymorphism of yaml. However this proved problematic as during unmarshaling the librarby did not know what to instantiate.
After trying to fix this a different approach was taken that ended up being a lot simpler.

\section{Loading}
When a load instruction is given to the system a call is done to LoadTemplate with the name of the template to be loaded.
This function loads the requested template and all other templates needed to satisfy its parents.
If a template could not be found this function will return nothing and the program will print an informational message that the template could not be loaded.
If the template depends on a parent template and this template could not be found an error condition is reported.
This is done because the program only allows a template deletion when it does not have any other templates that depend on it.
Thus this condition can only if the templates stored on disk were corrupted or modified outside the program.
To limit the scope of the prototype this condition can not be recovered from and has to be manually resolved by removing the affected templates or editing the YAML files to get them back in the proper state.

\section{Hikvision}
The Hikvision api was implemented using the Hikvision ISTAPI documentation\cite{noauthor_intelligent_nodate}.
Two paramaters had to be implemented.
The first one was the motion detection sensitivity paramater.
The XML for this has been mapped in a struct and is configured by reading the object, editing it and writing it back.
Since the whole object is used it is trivial to implement more of these settings.
For the purposes of this prototype only the sensitiviy parameter was used.
TBD: Insert example XML from camera
%TODO insert dummy XML payload

In order to compare the changes to the template the API implementation uses the three steps as follows.
First the prepare step must be executed whenever an action using the API is to be done.
This has been split since both the compare and the configure step would have to do the same actions of getting the information from the camera bbefore doing further operations.
After the prepare step has run the results are stored in the APIData field of the camera being operated on for use by the other functions.
A choice can be made to do either the compare or the configure step.
When a compare is done the XML is examined and their representations as paramaeters is constructed.
These values are then compared with the values from the template to detect any mismatches.
If a mismatch has been detected the parameter is added to a list so the program can output the expected and the actual value of that parameter.

The configure step also depends on the data gathered using the prepare step and modifies the downloaded XML with the values from the template.
By using the GetParam function of a template the template will provide a pointer to the proper parameter as described in the parameter enumeration part.
The parameter provides the value that should be input to the XML body and when all paramaters that the API supports have been updated the configuration is written back to the camera as documented in the manufacturer API documentation.

If the camera reports an error condition during the configuration process this error condition is reported backt to the user.

\section{VCA}
At the time of writing an implementation that can interact with a VCA camera has not been completed yet.
This will be part of the final portfolio.

%
%System test 1:
%blablabla
%Requirements:
%Results:

\chapter{Testing}
In order to verify the correct operation of the system it has been subjected to automatic unit tests and manual system tests.
The unit tests are automatically executed using the Go testing framework each time a commit is made to the git repository
As stated in the continuous integration section are a requirement before a pull request can be merged.
While the implementation of the prototype is not fully complete at the time of writing the expectation is that all unit tests pass at the end of the project and at that time a coverage report will be shown here indicating what is and isn't covered by the unit tests.

\section{System test}
System tests will list what requirements they test, what the test procedure is and what the result was.
These tests are used to determine if all requirements have been satisfied.
These system tests will be done in the last week to most closely resemble the state of the system.

System tests will be defined with the following information: a short name indicating the purpose of the test, a test description, a list of requirements being tested, and the results of the test.

\chapter{Conclusion}
\chapter{Recommendations}
If the configurable parameters grow in numbers it would be nice to separate them in different categories that can be folded.

Use git internal datastructure only

\printbibliography[heading=bibintoc]
\appendix
\chapter{Class diagram}
\begin{figure}[h!]
	\centering
	\includegraphics[scale=1]{class_diagram}
	\caption{Class diagram}
	\label{fig:class_diagram}
\end{figure}


% TODO: Remove this
%\chapter{Research methodology} \label{methods}
%Research methods are used according to the  In this chapter the methods that
%will be used are described and their application will be handled in chapter \ref{activities}.
%\section{Library}
%Library research is done to explore what is already done and what guidelines and theories exist that could help you further your design. Since the advent of the internet library research is also called desk research.
%
%\subsection{Available product analysis}
%An available product analysis is the process of finding out if there are already existing solutions to the problem at hand.
%
%\subsection{Best good and bad practices}
%The best good and bad practices is a research method where work of others is investigated and a way is found to incorporate this in your own research.
%
%\subsection{Literature study}
%Literature study is used to gather information about a subject, determining what material to read in detail and summarising your findings.
%
%\subsection{Design pattern research}
%Design patterns are documented solutions to frequently encountered problems or challenges in software engineering; they incorporate good software engineering principles. Having good knowledge of these patterns can help improve the quality of designed software.
%
%\section{Workshop}
%Workshop research is done to explore opportunities. Prototyping, designing and co-creation activities are all ways to gain insights in what is possible and how things could work.
%
%\subsection{Prototyping}
%With this domain a prototype is built to show an idea to stakeholder and to evaluate if it is worth expanding upon.
%
%\subsection{Requirements Prioritization}
%This method can be used to gather requirements from stakeholders and will be used during the design phase of the project. Once the requirements have been determined a prioritization can be made so that they can be implemented based on priority.
%
%\subsection{IT architecture sketching}
%Come together around a whiteboard and draw the high-level architecture based upon discussion before and during the drawing process. Stay away from details, unless they are important for the overall design.
%
%\subsection{Unit test}
%Define one or more tests for each ‘atomic part’ of the code (e.g. a method or function). The unit should be tested in isolation.
%
%\chapter{Activities} \label{activities}
%The project is broken down into four phases consisting of the Definition, Analysis, Design and Implementation phase. Each phase is then further broken down into
%research activities as described in section \ref{methods}.
%
%\section{Project definition}
%\subsection{Available product analysis}
%During the definition phase of the project an available product analysis was done to evaluate relevant research done in the past that may be of relevance to this
%project. During the analysis a thesis \cite{kesteloo_onvif_2016} was found that completed a lot of work that would have been done in the original scope of the project. After
%this analysis the decision was made to move the scope of the project to prevent research from unnecessarily being repeated. With the new scope in mind the analysis was
%conducted again and some solutions were found that could get and set settings from an ONVIF enabled camera but none were found that could create a generic configuration nor
%were there generic solutions for cameras not implementing ONVIF.
%
%\section{Analysis}
%\subsection{Best good and bad practices}
%The cameras that will be used implement multiple APIs, one more useful than the other. Before the design can be made it should be evaluated which API will be used and
%how it works. The implementation of this API can have effect on certain parts of the design in regards to the chosen programming language, interface design and
%implemented capabilities.
%
%\subsection{Prototyping}
%In order to gain a better understanding of the camera APIs some very simple proof of concepts can be built to gain experience with the API. This knowledge can then be applied during the design and implementation phases.
%
%\section{Design}
%\subsection{Design pattern research}
%The prototype will include an interface between a generic configuration and the camera specific configurations. In order to make this interface future proof, design patterns should be researched so that the interface can be designed to be easily extended.
%
%\subsection{IT architecture sketching}
%During the design phase thought will be put into the architecture of the prototype. This will start with a sketch of the overall prototype and this will be refined into a functional and technical design.
%
%\section{Implementation}
%\subsection{Prototyping}
%The functional and technical design will be implemented in the form of a prototype.
%
%\subsection{Unit test}
%To validate the operation of the prototype unit tests will be written to cover as many code paths as possible. Next to that an integration test will be done to validate
%proper functioning of the prototype. The details of this test will be formalized at a later stage in a test plan.
%
%\subsection{Code review}
%In order to maintain the quality of the prototype, code reviews will be done on all pull requests that are to be included in the prototype. The review can be done by either the company mentor or another team member.
%
%\chapter{Project management}
%The project will be conducted using Scrum. Scrum is already used within the team at BTG and it was decided that the project would use the same process to easily communicate progress and impediments to the team.
%
%Sprints will be set at two weeks and start after the sprint planning on Monday. Scrum activities will be done together with BTG's team when this makes sense. Story mapping, backlog refinement, sprint planning and the retrospective will not be integrated with the team as this would take time away from both the graduate student and the team, as their work is not directly related to each other nor would they have much to contribute to the other party during those sessions. The only persons in these meetings will be the graduate student and his mentor. Effectively this means that only the daily stand-up will be part of the team's stand-up so that the graduate student can share what he is working on and indicate if there are any impediments that the team can help resolve.
%
%BTG has a separate environment for students in their systems. This way they can keep their own backlog without mixing in with work from other teams. Next to that they also have their own place to host documentation and keep their code. In case a student needs to have access to internal resources these can be made available on an as needed basis.
%
%\chapter{Stakeholder analysis}
%During the definition phase a couple of stakeholders were identified. They were identified based on the fact that the outcome of the project would have
%some sort of effect on them. Thymo is conducting the project and has the biggest stake in it. Wouter and Silke are both software engineers within the
%back-end team and after the project is finished might have to work with the resulting project. Silke in particular has indicated that he would be helped
%by the outcome as he currently has to spend a lot of time of batching API calls through a script if a larger amount of cameras need to be configured.
%Alex is a Product Manager within BauWatch that had some ideas about the project objective.
%
%\begin{table}[h]
%    \centering
%    \caption{Stakeholders}
%    \label{tab:stakeholders}
%    \begin{tabular}{ | m{10em} | m{10em} | }
%    \hline
%    \textbf{Stakeholder} & \textbf{Type} \\
%    \hline
%    Thymo van Beers & Key figure \\
%    \hline
%    Wouter Horlings & Influencer
%    \\ \hline
%    Silke Hofstra & Influencer
%    \\ \hline
%    Alex van der Leij & Onlooker
%    \\ \hline
%    \end{tabular}
%\end{table}
 
 %%%%%%%% Plan of Approach ends here!!!! %%%%%%
 %%%%%%%%%%%%%%%%%%%%%%%%%%%%%%%%%%%%%%%%%%%%%%%

% TODO Replace with bibtex in template
%\begin{thebibliography}{2}
%    \bibitem{tim_k_onvif}
%        ``ONVIF Plug-in", \textit{Tim Kesteloo}, [online]. Available: \url{https://hbo-kennisbank.nl/details/sharekit_hh:oai:surfsharekit.nl:15d206a2-dc4e-4dbd-9719-339793255ddd}. [Accessed: 01-12-2021]
%    \bibitem{dot_framework}
%        ``The DOT-Framework", \textit{HBO-i}, [online]. Available: \url{https://ictresearchmethods.nl/The_DOT_Framework}. [Accessed: 01-12-2021]
%    \bibitem{ti_wmc}
%        ``Twente Institute for Wireless and Mobile Communications", \textit{SALUS}, [online]. Available:
%        \url{https://www.sec-salus.eu/consortium/ti-wmc/}. [Accessed: 20-01-2022]
%	\bibitem{what_is_foreman}
%		``Foreman :: Introduction", \textit{Foreman authors}, [online]. Available:
%		\url{https://www.theforeman.org/introduction.html}. [Accessed: 16-02-2022]
%	\bibitem{foreman_hostgroups}
%		``Managing Hosts", \textit{Foreman authors}, [online]. Available:
%		\url{https://docs.theforeman.org/3.1/Managing_Hosts/index-foreman-el.html}. [Accessed: 16-02-2022]
%    % \bibitem{moscow}
%    % ``MoSCoW Analysis (6.1.5.2)". A Guide to the Business Analysis Body of Knowledge (2 ed.).
%    % International Institute of Business Analysis. 2009. ISBN 978-0-9811292-1-1
%\end{thebibliography}


\end{document}
