\chapter{Conclusion}
%Questions
%
%Subquestions:
%	1. What parameters are of interest to BauWatch?
%		List them?
%		Interview
%		Through interview determined X and Y were relevant and have been implemented in the proof of concept.
%	2. How should a configuration be represented to express settings for different cameras?
%	3. 
%Table of achieved requirements shows musts and should have been achieved...

BauWatch currently has no proper management solution to control its camera configurations.
To find a solution to this problem this thesis answered the question: ,,How can a large number of remote security cameras of different models and brands be managed from a centralized system?". This has been done by doing research on already existing solutions, creating a design based on collected requirements and implementing this as a proof of concept

Before the design could be made it was important to know what kind of parameters are relevant to BauWatch.
During research it was determined that parameters relating to automatic camera focus and zoom, video quality, network configuration and detection sensitivity were most relevant and the system has been designed in such a way that these parameters can be supported.

Based on the conducted research and the collected requirements a design of a centralized parameter system with the ability of expressing settings for different cameras has been made.

This design has then be implemented in a proof of concept of which all must and should requirements have been met.

With the implemented proof of concept a method for managing a large number of remote security cameras of different models and brands has been provided which could be further developed into a fully functioning product.

%In order to prove the validity of the design two of these parameters have been included in the proof of concept.
%
%The third question that was posed was how a configuration for a camera should be represented to express settings for different cameras.
%Using the design it can be concluded that by providing a conversion layer between a template and CameraAPI, parameters only need to express a desired situation instead of an absolute value. 
%
%%%%%%%%%%%
%%To gain a better understanding of what these camera configurations looked like, the first subquestion: ,,What parameters are of interest to BauWatch?" was answered through an interview with product management.
%%In this interview it became clear that parameters  were interesting and that the design should be able to support these kinds of parameters.
%%
%%With these parameters in mind a design of the system architecture could be made as been presented in chapter \ref{sec:design} the other subquestions could be answered in the form of the design as seen in chapter \ref{sec:design} and the proof of concept.
%%
%%To be able to properly address this question it has been divided into three subquestions.
%%First of all a better understanding of 
%%The first question is ,What parameters are of interest of BauWatch?"
%%After conducting an interview with product management it was clear that 
%%settings that did not change often.
%%
%%Because
%%
%%In order to solve this problem an analysis of requirements was done which led to research being conducted to determine if there were any existing solutions, a design being made and a prototype that was implemented.
%%
%%The question this thesis answered is: 
%
%From the conducted tests it can be concluded that all requirements marked as must and should have been met.
%
%Some recommendations have been made for future studies or development in the next chapter.

%%%



\chapter{Recommendations}
\label{sec:recommendations}
Based on the results gathered by this project it is recommended that BauWatch further develops the proposed design into a fully functioning product.
During the implementation of the prototype some aspects were discovered that were not implemented during this project but are important to considering when doing further research on this topic.

\section{User interface}
Because the originally envisioned web interface and REST API implementation had to be replaced with a simple command line interface due to time constraints it is recommended that a replacement user interface is developed.
This would best be done as the originally intended web interface using the provided wireframes and REST API design as input.

\section{Extra parameters}
The prototype only contains two parameters (at the time of writing the second parameter is not quite working yet) to demonstrate the concept of layering these using templates and their use in a generic way.
In order to add more functionality to the system the implementation of the camera APIs should be extended to allow more camera parameters to be configured.
When this is done it is also good to implement more parameter types for the templates.
While there are two generic parameter types for working with string and integer values it would be better to implement a parameter type specifically designed for the type of setting it controls.
This way the template parameter can have behavior that makes it easier to configure for a range of cameras or allows more fidelity for a specific parameter.
Since there is no restriction to the operation of a parameter type other than conforming to the parameter interface quite complex types could be implemented.
An example of a parameter type to implement could be to input the times a camera is set to an armed state to actively monitor for events using a menu that allows the user to select a certain time period.

\section{Testing}
While the templates are covered by unit tests it was not possible to implement automatic tests for the camera API within the given time frame.
This could be improved by using the Go httptest package \cite{noauthor_golang_nodate} to implement a HTTP server that acts like a camera but responds with a predetermined set of data.
This way one can be sure that tests can be reliably conducted without depending on a camera being available for testing and that there were no external changes that could affect the behavior of the program.
Of course this means that in the event something changes on the camera side, like a firmware update or undocumented change in a model, that extra tests are created to reflect the changed situation.

\chapter{Reflection}
\label{reflection}
At the start of my graduation period halfway through November, the winter season was approaching and COVID-19 infection rates were again going up.
As a result the country entered into another lockdown in an attempt to flatten the curve.
This unfortunately meant that working from home was again in effect.
This meant I was not able to meet a lot of people at the start of my internship and it took a while to get to know my colleagues.
While I was able to mostly work from the office in the first two weeks to get set up with the proper materials and to get to know my mentor I did not get to meet all of my team members until three months after starting.

While preparing the plan of approach some preliminary library research was done before fully digging into the research phase of the project.
During this period a previous thesis on creating an ,,ONVIF Plug-in" \cite{kesteloo_onvif_2016} was found.
This thesis was found to contain a lot of material that would have been produced within the original scope of the project.
Originally the project was aimed at creating software that could change camera settings through ONVIF to allow universal control.
After consulting with my mentor, graduation teacher and the graduation coordinator it was decided that the project could continue without delaying the project and going through the formalities of defining a new project if a adjusted proposal could be provided.
Unfortunately redefining the project meant that the plan of approach had to be adjusted with new research objectives and problem description and with some adjustments this was finally approved after the Christmas break in the seventh week after the starting date.

It was recommended to me to implement the prototype in Go.
While I never programmed in Go and had to learn it from scratch I was interested in the language and since a lot of knowledge was present at BauWatch I did not want to dismiss it right away.
While my lack of experience did sometimes briefly impede my progress in order to grasp a new concept or to refactor some things because it ended up not being a good approach in Go, I did learn a lot about the language and got to appreciate it as a valuable tool to add to my knowledge.

%covid
%onvif
%redefine project
%plan of approach
%new programming language Go
%what went well
%what went bad
%lack of other interns
%small company
%challenging
%setbacks
