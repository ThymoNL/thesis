\chapter{Conclusion}
Unfortunately there was no time to implement all requirements.
Nevertheless there is enough evidence that supports the use of a generic system for configuring cameras.

\chapter{Recommendations}
If the configurable parameters grow in numbers it would be nice to separate them in different categories that can be folded.

Use git internal datastructure only

\chapter{Reflection}
At the start of my graduation period halfway through november, the winter season was approaching and COVID-19 infection rates were again going up.
As a result the country entered into another lockdown in an attempt to flatten the curve.
This unfortunately meant that working from home was again in effect.
This meant I was not able to meet a lot of people at the start of my internship and it took a while to get to know my colleagues.
While I was able to mostly work from the office in the first two weeks to get set up with the proper materials and to get to know my mentor I did not get to meet all of my team members until three months after starting.

While preparing the plan of approach some preliminary library research was done before fully digging into the research phase of the project.
During this period a previous thesis on creating an ,,ONVIF Plug-in" \cite{kesteloo_onvif_2016} was found.
This thesis was found to contain a lot of material that would have been produced within the original scope of the project.
Originally the project was aimed at creating software that could change camera settings through ONVIF to allow universal control.
After consulting with my mentor, graduation teacher and the graduation coordinator it was decided that the project could continue without delaying the project and going through the formalities of defining a new project if a adjusted proposal could be provided.
Unfortunately redefining the project meant that the plan of approach had to be adjusted with new research objectives and problem description and with some adjustments this was finally approved after the christmas break in the seventh week after the starting date.

It was recommended to me to implement the prototype in Go.
While I never programmed in Go and had to learn it from scratch I was interested in the language and since a lot of knowledge was present at BauWatch I did not want to dismiss it right away.
While my lack of experience did sometimes briefly impede my progress in order to grasp a new concept or to refactor some things because it did not work well in Go I did learn a lot about the language and got to appreciate it as a valuable tool to add to my repetoire.


covid
onvif
redefine project
plan of approach
new programming language Go
what went well
what went bad
lack of other interns
small company

